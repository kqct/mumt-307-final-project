% Options for packages loaded elsewhere
\PassOptionsToPackage{unicode}{hyperref}
\PassOptionsToPackage{hyphens}{url}
%
\documentclass[
  ignorenonframetext,
]{beamer}
\usepackage{pgfpages}
\setbeamertemplate{caption}[numbered]
\setbeamertemplate{caption label separator}{: }
\setbeamercolor{caption name}{fg=normal text.fg}
\beamertemplatenavigationsymbolsempty
% Prevent slide breaks in the middle of a paragraph
\widowpenalties 1 10000
\raggedbottom
\setbeamertemplate{part page}{
  \centering
  \begin{beamercolorbox}[sep=16pt,center]{part title}
    \usebeamerfont{part title}\insertpart\par
  \end{beamercolorbox}
}
\setbeamertemplate{section page}{
  \centering
  \begin{beamercolorbox}[sep=12pt,center]{part title}
    \usebeamerfont{section title}\insertsection\par
  \end{beamercolorbox}
}
\setbeamertemplate{subsection page}{
  \centering
  \begin{beamercolorbox}[sep=8pt,center]{part title}
    \usebeamerfont{subsection title}\insertsubsection\par
  \end{beamercolorbox}
}
\AtBeginPart{
  \frame{\partpage}
}
\AtBeginSection{
  \ifbibliography
  \else
    \frame{\sectionpage}
  \fi
}
\AtBeginSubsection{
  \frame{\subsectionpage}
}
\usepackage{amsmath,amssymb}
\usepackage{iftex}
\ifPDFTeX
  \usepackage[T1]{fontenc}
  \usepackage[utf8]{inputenc}
  \usepackage{textcomp} % provide euro and other symbols
\else % if luatex or xetex
  \usepackage{unicode-math} % this also loads fontspec
  \defaultfontfeatures{Scale=MatchLowercase}
  \defaultfontfeatures[\rmfamily]{Ligatures=TeX,Scale=1}
\fi
\usepackage{lmodern}
\ifPDFTeX\else
  % xetex/luatex font selection
\fi
% Use upquote if available, for straight quotes in verbatim environments
\IfFileExists{upquote.sty}{\usepackage{upquote}}{}
\IfFileExists{microtype.sty}{% use microtype if available
  \usepackage[]{microtype}
  \UseMicrotypeSet[protrusion]{basicmath} % disable protrusion for tt fonts
}{}
\makeatletter
\@ifundefined{KOMAClassName}{% if non-KOMA class
  \IfFileExists{parskip.sty}{%
    \usepackage{parskip}
  }{% else
    \setlength{\parindent}{0pt}
    \setlength{\parskip}{6pt plus 2pt minus 1pt}}
}{% if KOMA class
  \KOMAoptions{parskip=half}}
\makeatother
\usepackage{xcolor}
\newif\ifbibliography
\setlength{\emergencystretch}{3em} % prevent overfull lines
\providecommand{\tightlist}{%
  \setlength{\itemsep}{0pt}\setlength{\parskip}{0pt}}
\setcounter{secnumdepth}{-\maxdimen} % remove section numbering
\ifLuaTeX
  \usepackage{selnolig}  % disable illegal ligatures
\fi
\IfFileExists{bookmark.sty}{\usepackage{bookmark}}{\usepackage{hyperref}}
\IfFileExists{xurl.sty}{\usepackage{xurl}}{} % add URL line breaks if available
\urlstyle{same}
\hypersetup{
  hidelinks,
  pdfcreator={LaTeX via pandoc}}

\author{}
\date{}

\begin{document}

\hypertarget{mumt-307-semester-project}{%
\section{MUMT 307 Semester Project}\label{mumt-307-semester-project}}

\hypertarget{implementation-of-basic-wavetabledelay-lineetc.-in-rust-too-software-focused}{%
\subsection{Implementation of Basic Wavetable/Delay Line/etc. in Rust?
(too software
focused?)}\label{implementation-of-basic-wavetabledelay-lineetc.-in-rust-too-software-focused}}

\hypertarget{something-to-do-with-amfm-synthesis-will-we-cover-this}{%
\subsection{something to do with AM/FM synthesis? (will we cover
this?)}\label{something-to-do-with-amfm-synthesis-will-we-cover-this}}

\hypertarget{a-granular-synthesizer-in-maxc-scope}{%
\subsection{a granular synthesizer in Max/C
(scope?)}\label{a-granular-synthesizer-in-maxc-scope}}

\hypertarget{avoiding-matlab-this-isnt-a-project}{%
\subsection{avoiding MATLAB (this isn\textquotesingle t a
project)}\label{avoiding-matlab-this-isnt-a-project}}

\hypertarget{waveform-synthesis-i-saw-a-video-on-the-nes-sound-chip-i-know-we-said-were-skipping-this}{%
\subsection{waveform synthesis? I saw a video on the NES sound chip (i
know we said we\textquotesingle re skipping
this)}\label{waveform-synthesis-i-saw-a-video-on-the-nes-sound-chip-i-know-we-said-were-skipping-this}}

\hypertarget{reimplementing-types-of-fourier-transforms-neat-learning-tool-weird-project}{%
\subsection{reimplementing types of Fourier transforms (neat learning
tool, weird
project?)}\label{reimplementing-types-of-fourier-transforms-neat-learning-tool-weird-project}}

\end{document}
